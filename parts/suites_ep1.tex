\documentclass[../main.tex]{subfiles}

\begin{document}

\onlyinsubfile{
    \begin{center}
    {\LARGE \textsc{Compléments:\\ Suites, épisode 1}}
    \end{center}
}

    Ce cours a pour objet d'approfondir les notions vues au programme de Terminale sur l'étude des suites numériques: monotonie, <<bornitude>>, convergence, et d'en introduire des aspects allant au-delà. On s'intéressera ici à l'étude des suites récurrentes d'ordre un, dans un contexte plus général que ce que vous avez pu rencontrer au détour d'un exercice, et on introduira la notion de série numérique.


\section{Suites arithmético-géométriques}

On commence par étudier le cas particulier des suites récurrentes vérifiant une relation affine $u_{n+1}=au_n+b$. Vous connaissez les cas $a=1$ et $b=0$.

\begin{mydef}
    Soit $(u_n)_{n\in\N}$ une suite réelle. On dit que $u$ est \textit{arithmético-géométrique} si et seulement s'il existe deux réels $a$ et $b$, avec $a\neq 1$, tels que
    \[
    \forall n\in\N\quad
    u_{n+1} = au_n + b.
    \]
\end{mydef}

\begin{thm}[Expression des suites arithmético-géométriques]\label{thmAG}
Soit $(u_n)_{n\in\N}$ une suite arithmético-géométrique de paramètres $a$ et $b$. Soit $c$ l'unique réel vérifiant $c=ac+b$. Alors:
    \[
    \forall n\in\N\quad
    u_n = (u_0-c)a^n + c.
    \]
\end{thm}

\begin{exo}[F]
    Soit $u$ une suite arithmético-géométrique. Dans chacun des cas suivants, déterminer son expression en fonction de $n$, étudier sa convergence:
    \begin{multicols}{2}
    \begin{enumerate}
        \item \[\begin{cases}
            u_0 = 1 \\
            u_{n+1} = \frac{1}{2}u_n - 1
        \end{cases}\]
        \item \[\begin{cases}
            u_0 = 0 \\
            u_{n+1} = -3u_n + 5
        \end{cases}
        \]
        \item \[\begin{cases}
            u_0 = 5 \\
            u_{n+1} = \frac{1}{5}u_n + 2
        \end{cases}
        \]
        \item \[\begin{cases}
            u_0 = -4 \\
            u_{n+1} = \frac{5}{4}u_n + 2
        \end{cases}
        \]
    \end{enumerate}
    \end{multicols}
\end{exo}

\begin{exo}[M]
    Démontrer le théorème \ref{thmAG}.
\end{exo}


\section{Suites récurrentes d'ordre un}

\subsection{Généralités}

\begin{mydef}[Suite récurrente d'ordre un]
    Une suite $(u_n)_{n\in\N}$ est dite \textit{récurrente d'ordre un} s'il existe une fonction réelle $f:A\subset\R \rightarrow\R$, telle que
    \[
    \forall n\in\N
    \quad u_{n+1} = f(u_n). \]
\end{mydef}

\begin{exe}\label{réc1}
    La suite réelle définie par $u_0=2$ et $u_{n+1}=\frac{1}{2}{u_n}^2+u_n+1$ est une suite récurrente d'ordre un ; quelle est la fonction $f$ correspondante ?
\end{exe}

Ce type de problème rentre dans le domaine d'étude des \textit{systèmes dynamiques}, fondamental en mathématiques pures comme appliquées. Les systèmes dynamiques interviennent partout où l'on veut décrire l'évolution d'un paramètre dans le temps, dès que les valeurs successives de ce paramètre ont une dépendance entre elles.

Pour des fonctions $f$ assez moches, on ne peut pas exprimer $u_n$ en fonction de $n$, même en utilisant des astuces classiques (composer par des fonctions usuelles bien choisies...). Et pourtant, des fonctions $f$ plutôt moches interviennent dans la plupart des systèmes dynamiques étudiés.

L'étude des solutions au problème $u_{n+1}=f(u_n)$ sera donc qualitative. Il faut vérifier la bonne définition de la suite à tout rang, voir si elle est monotone, si elle est bornée, si elle converge, quelle pourrait être sa limite, en discutant selon les valeurs de son premier terme $u_0$.

\begin{rem}
    Il \textbf{faut} vérifier que la suite $(u_n)$ est bien définie, c'est-à-dire que $u_n\in A$ pour tout $n$. En effet pour $u_0=1$ et $f$ définie sur $[0,+\infty[$ par $f(x)=\sqrt{x}-1$, la suite de relation de récurrence $u_{n+1}=f(u_n)$ n'est pas définie à partir du rang 2...
\end{rem}

\begin{prop}\label{wellDef}
    Si $A$ est stable par $f$ (i.e. $f(A)\subset A$), alors $u$ est bien définie.
\end{prop}

\begin{exo}[M]
	Démontrer la proposition \ref{wellDef}.
\end{exo}

\subsection{Monotonie}

En général, les deux méthodes suivantes permettent de traiter la plupart des suites récurrentes d'ordre un:
\begin{itemize}
    \item étudier le signe de $u_{n+1}-u_n=f(u_n)-u_n$, en étudiant celui de la fonction $g:x\mapsto f(x) - x$. 
    \item étudier les variations de $f$ : si $f$ est croissante sur $A$, on compare $u_0$ et $u_1$. Si on a $u_0\leq u_1=f(u_0)$, une récurrence donne $u_n\leq u_{n+1}=f(u_n)$ par croissance de $f$. Si on a $u_1\leq u_0$ alors par récurrence $u_{n+1}\leq u_n$.
\end{itemize}
Bien sûr, le sens de variation de $f$ ou le signe de $g$ peuvent changer sur $A$. Dans ce cas, il faut chercher un intervalle $I\subset A$ sur lequel $g$ est de signe constant ou $f$ ne change pas de sens de variation, et qui vérifierait $u_n\in I\quad \forall n\in\N$.


\subsection{Convergence}

\begin{prop}
    On suppose que $f$ est continue et que $u$ converge. Soit $\ell$ sa limite. 
    \begin{itemize}
        \item Si $\ell$ est dans $A$ alors par continuité de $f$ on a $f(\ell)=\ell$.
        \item Sinon, $\ell$ est un bord de $A$ mais n'y appartient pas (comme $0$ avec $]0,1]$).
    \end{itemize}
\end{prop}

\begin{rem}
    Bien évidemment, si $\ell\in \R/A$ mais $f$ admet une limite en $\ell$, on prolonge $f$ en $\ell$ en posant $f(\ell)$ égal à cette limite. La fonction obtenue est continue sur $A\cup\{\ell\}$ et on se ramène au cas précédent.
\end{rem}

\begin{exo}[M]
	La suite de l'exemple \ref{réc1} est-elle monotone ? Converge-t-elle ?
\end{exo}

\begin{exo}[D]
	On considère la suite $(u_n)$ de premier terme un entier $u_0\geq 2$ et vérifiant la relation de récurrence \[u_{n+1}=\frac{{u_n}^2+2}{2u_n}.\] 
	\begin{enumerate}
		\item Montrer par récurrence que c'est une suite de nombres rationnels.
		\item Montrer que la demi-droite $[\sqrt{2},+\infty[$ est stable par la fonction $f:x\longmapsto \dfrac{x^2 + 2}{2x}$.
		\item Étudier le signe de $g:x\longmapsto f(x)-x$ ; en déduire que $(u_n)$ est décroissante.
		\item Montrer qu'elle converge vers $\sqrt{2}$.
	\end{enumerate}
\end{exo}


\begin{exo}[D]
	Soit $f:x\in\R \longmapsto (x-1)^3$. On cherche à étudier les suites de la forme $u_{n+1}=f(u_n)$, où $u_0\leq 1$.\begin{enumerate}
		\item Montrer que l'intervalle $]-\infty,1]$ est stable par $f$.
		\item Étudier le signe de $g:x\longmapsto f(x)-x$ sur $\R$. En conclure que $(u_n)$ est strictement décroissante.
		\item Montrer que $u_n\xrightarrow[n\to+\infty]{}-\infty$.
	\end{enumerate}
\end{exo}

\begin{exo}[D]
	Soit $u$ la suite récurrente définie par $u_0<2$ et $u_{n+1} = -({u_n}-2)^2 + 2$. Étudier sa convergence.
\end{exo}

\end{document}