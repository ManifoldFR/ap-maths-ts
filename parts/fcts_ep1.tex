\documentclass[../main.tex]{subfiles}

\begin{document}

\onlyinsubfile{
    \begin{center}
    {\LARGE \textsc{Compléments:\\ Fonctions de la variable réelle, épisode 1}}
    \end{center}
}

    Dans ce cours, nous approfondirons les notions sur les fonctions de la variable réelle: limites, continuité, dérivabilité. On passera en revue des notions sur l'ensemble $\R$ des nombres réels, avant de voir quelques résultats sur les limites et les fonctions continues.


\section{Topologie de $\R$}

\subsection{Borne supérieure}

\begin{mydef}[Majorant, minorant]
Soit $X$ une partie de $\R$. On dit que $X$ est\begin{enumerate}
\item \textit{majorée} s'il existe un réel $M$ tel que pour tout $x\in X$, $x\leq M$ ; un tel $M$ est alors appelé \textit{majorant de $X$}
\item \textit{minorée} s'il existe un réel $m$ tel que pour tout $x\in X$, $m \leq x$ ; un tel $m$ est alors appelé \textit{minorant de $X$}.
\end{enumerate}
\end{mydef}

\begin{rem}
S'il existe un minorant ou un majorant, celui-ci n'est \textbf{jamais} unique : en effet, si $M$ est un majorant, $M+1$ aussi, si $m$ est un minorant, $m-1$ aussi...
\end{rem}

\begin{mydef}[Borne supérieure]\label{defsup}
    Soit $X$ une partie de $\R$. On appelle \textit{borne supérieure de $X$}, et on note $\sup X$, le plus petit majorant de $X$, s'il existe. La borne supérieure est caractérisée par les deux propriétés suivantes: $\alpha$ est la borne supérieure de $X$ si et seulement si :\begin{enumerate}
        \item c'est un majorant de $X$: pour tout $x$ de $X$, $x\leq \alpha$.
        \item pour tout majorant $M$ de $X$, on a $\sup X\leq M$.
    \end{enumerate}
\end{mydef}

\begin{thm}[dit de la borne supérieure]
    Toute partie \textbf{non vide} et \textbf{majorée} de $\R$ admet une borne supérieure.
\end{thm}

\begin{exo}[F]
    Déterminer les bornes supérieures des ensembles suivants:
    \begin{multicols}{2}
    \begin{enumerate}
        \item $\R$
        \item $[0,1[$
        \item $\{n\in\Z,\ n\text{ est pair}\}$
        \item $]-\infty, 0]$
        \item $\varnothing$
        \item $\left\{1-\dfrac{1}{n},\ n\in\N^*\right\}$
    \end{enumerate}
    \end{multicols}
\end{exo}

\begin{prop}[Autre caractérisation de la borne supérieure] \label{caractsup}
    La propriété 2 de la définition \ref{defsup} peut être remplacée par l'énoncé suivant:
    
    \textup{pour tout $\epsilon > 0$, il existe $x\in X$ tel que $x>\alpha - \epsilon$.}
\end{prop}

\begin{cor}
    Soit $A\subset\R$, non vide et majoré, et $\alpha \coloneqq \sup X$. Alors, il existe une suite $(u_n)$ à valeurs dans $X$ et croissante telle que $u_n\xrightarrow[n\to+\infty]{}\alpha$.
\end{cor}

\begin{rem}
	La réciproque est fausse.
\end{rem}

\begin{exo}[M]
    Démontrer la proposition \ref{caractsup} et son corollaire.
\end{exo}

\begin{mydef}[Borne inférieure]\label{definf}
	On définit de même la \textit{borne inférieure} d'une partie $X$ de $\R$, notée $\inf X$, comme le plus grand de ses minorants, s'il existe. Elle est caractérisée de la même façon: un réel $\alpha$ est la borne inférieure de $X$ si et seulement si:\begin{enumerate}
		\item c'est un minorant de $X$ : pour tout $x\in X$, $\alpha \leq x$.
		\item pour tout minorant $m$ de $X$, on a $m\leq\inf X$.
	\end{enumerate}
	et la deuxième condition équivaut à : pour tout $\epsilon>0$, il existe $x\in X$ tel que $x<\alpha + \epsilon$.
\end{mydef}

On a de même l'existence d'une suite $(u_n)$ décroissante à valeurs dans $X$ telle que $u_n\longrightarrow \alpha$.

\subsection{Parties ouvertes et fermées de $\R$}

\begin{mydef}[Ouvert]
	Soit $U\subset\R$. On dit $U$ est \textit{ouvert} si pour tout $x\in U$, il existe $\epsilon > 0$ tel que ${]x-\epsilon,x+\epsilon[}\subset U$.
\end{mydef}

\begin{prop}\leavevmode
	\begin{itemize}
		\item Une réunion quelconque d'ouverts est un ouvert.
		\item Une intersection \textbf{finie} d'ouverts est un ouvert.
	\end{itemize}
\end{prop}

\begin{exo}[M]
	Le démontrer. Trouver un contre-exemple pour une intersection \textit{infinie} d'ouverts.
\end{exo}

\begin{exe}
	\begin{multicols}{2}
		\begin{enumerate}
			\item $\R$
			\item $\varnothing$
			\item $]0,1[$
			\item $]a,+\infty[$ où $a\in\R$
			\item $\R^*$
			\item ${]-2,-1[}\cup{]0,1[}$
		\end{enumerate}
	\end{multicols}
\end{exe}

\begin{mydef}[Fermé]
	Soit $F\subset\R$. On dit que $F$ est \textit{fermé} si son complémentaire $\R/F$ est un ouvert de $\R$.
\end{mydef}

\begin{prop}\leavevmode
	\begin{itemize}
		\item Une intersection quelconque de fermés est un fermé.
		\item Une réunion \textbf{finie} de fermés est un fermé.
	\end{itemize}
\end{prop}

\begin{exo}[M]
	Le démontrer. Trouver un contre-exemple pour une réunion \textit{infinie} de fermés.
\end{exo}

\begin{exe}\leavevmode
	\begin{multicols}{2}
		\begin{enumerate}
			\item $\R$
			\item $\varnothing$
			\item $[0,+\infty[$
			\item $[0,1]$
			\item $\{0\} $
			\item L'ensemble des entiers pairs.
		\end{enumerate}
	\end{multicols}
\end{exe}

\begin{exo}[M]
	Montrer qu'une partie $F\subset\R$ est fermée si et seulement si pour toute suite $(u_n)$ d'éléments de $F$ convergeant vers un réel $\ell$, on a $\ell\in F$.
\end{exo}

\begin{mydef}[Point adhérent]
	Soit $X\subset\R$. On dit que $x\in\R$ est \textit{adhérent à $X$} s'il existe une suite $(u_n)$ de $X$ telle que $u_n\longrightarrow x$.
\end{mydef}

\begin{mydef}[Adhérence]
	On appelle \textit{adhérence} de la partie $X$ de $\R$, et on note $\overline{X}$ l'ensemble de ses points adhérents.
\end{mydef}

\begin{prop}\label{sader}
	Soient $X,Y\subset\R$. Alors:\begin{itemize}
		\item $X\subset\overline{X}$
		\item $X\subset Y\Longrightarrow \overline{X}\subset\overline{Y}$
		\item $\overline{X\cap Y}\subset \overline{X}\cap\overline{Y}$
		\item $\overline{X\cup Y} = \overline{X}\cup\overline{Y}$
	\end{itemize}
\end{prop}

\begin{exo}[M]
	Démontrer la proposition \ref{sader}.
\end{exo}


\subsection{Intervalles de $\R$}

\begin{mydef}[Segment]
    Soient $a$ et $b$ deux réels. On appelle \textit{segment d'extrémités $a$ et $b$}, et on note $[a,b]$, la partie de $\R$ définie par:
    \[
    [a,b] \coloneqq \{(1-t)a+tb,\quad t\in[0,1] \}.
    \]
\end{mydef}

\begin{exo}[F]
    On suppose que $a\leq b$. Montrer que
    \[
    [a,b] = \{x\in\R,\quad a\leq x\leq b \}.
    \]
\end{exo}

\begin{mydef}[Intervalle]
    On appelle \textit{intervalle} toute partie $I$ de $\R$ telle que pour tous $a,b\in I$, $[a,b]\subset I$. En d'autres termes, si on prend deux points de $I$, le segment entre les deux est toujours dans $I$.
\end{mydef}

\begin{prop}[Inventaire des intervalles de $\R$]
    Le tableau suivant décrit l'ensemble des intervalles de $\R$.
    \begin{center}
    \begin{tabular}{|c|c|} \hline
        Borné & Non borné \\ \hline
        $\varnothing$ & $\R$ \\ \hline
        Intervalle semi-ouvert $]a,b]$ ou $[a,b[$ & \\ \hline
        Segment $[a,b]$ & $]-\infty,b]$ et $[a,+\infty[$ \\ \hline
        Intervalle ouvert $]a,b[$ & $]a,+\infty[$ et $]-\infty,b[$ \\ \hline
    \end{tabular}
    \end{center}
\end{prop}

\begin{comment}
\subsection{Densité d'une partie de $\R$}

\begin{mydef}[Densité]
    On dit que $A$ est dense dans $\R$ si pour tous réels $x$ et $y$ il existe un $a\in A\cap {]x,y[}$.
\end{mydef}

\begin{thm}[Propriété d'archimède]
    Soient $x<y$ deux réels avec $x>0$. Alors il existe $n\in\N$ tel que $y\leq nx$.
\end{thm}

\begin{prop}
    L'ensemble $\Q$ des nombres rationnels est dense dans $\R$.
\end{prop}

\begin{exo}[D]
    Le démontrer.
\end{exo}
\end{comment}

\section{Limites, continuité}

\subsection{Définitions}

On commence par donner une caractérisation de la convergence d'une fonction en un point, qui sera très commode à utiliser. Dans toute la suite, $f$ est une fonction $A\subset\R\rightarrow\R$.

\begin{prop}\label{caractCVfct}
    Soient $a\in\bar{A}$ et $\ell\in\overline{\R}$. Alors s'équivalent:
    \begin{enumerate}
        \item $f(x)\xrightarrow[x\to a]{}\ell$
        \item pour toute suite $(x_n)$ à valeurs dans $A$ telle que $x_n\longrightarrow a$, on a $f(x_n)\longrightarrow \ell$.
    \end{enumerate}
\end{prop}

Ainsi, une fonction est continue en un point $a\in A$ si et seulement si pour toute suite $(x_n)$ de $A$ telle que $x_n\longrightarrow a$, on a $f(x_n)\longrightarrow f(a)$.

\subsection{Théorème des valeurs intermédiaires}

\begin{thm}[dit des valeurs intermédiaires]
    Soient $a<b$ dans $\R$ et $f$ une fonction continue de $[a,b]$ dans $\R$. On suppose que $f(a)<0$ et $f(b)>0$. Alors il existe $c\in{]a,b[}$ tel que $f(c)=0$.
\end{thm}

\begin{exo}[Démonstration, D]
	Démontrer le théorème en considérant l'ensemble $X$ des points $x\in[a,b]$ tels que $f(x)\leq 0$. On pourra remarquer qu'il est non vide puisque $f(a)<0$ et majoré par $b$.
\end{exo}

\begin{exo}[M]\leavevmode
	\begin{enumerate}
		\item Soit $f$ une fonction continue sur un intervalle $I$. Soient $a<b\in I$, et $k\in\R$. On suppose que $f(a)<k<f(b)$. Montrer qu'il existe $c\in{]a,b[}$ tel que $f(c)=k$.
		\item Montrer que $f(I)$ est un intervalle.
	\end{enumerate}
\end{exo}

\begin{exo}[M]
	Soit $f:[a,b]\longrightarrow[a,b]$ continue. Montrer qu'il existe $c\in[a,b]$ tel que $f(c)=c$ (on appelle un tel $c$ un \textit{point fixe} de $f$).\\
	\textit{Indication:} Considérer $g:x\longmapsto f(x)-x$.
\end{exo}

\subsection{Théorème de compacité}

\begin{thm}
	Soient $a<b$ dans $\R$ et $f:[a,b]\longrightarrow\R$ une fonction continue. Alors $f$ est bornée et admet un minimum et un maximum.
\end{thm}

\subsection{Théorème de la bijection continue}

\begin{mydef}
Soient $A$ et $B$ deux parties de $\R$. On dit qu'une fonction $f:A\longrightarrow B$ est une \textit{bijection} de $A$ vers $B$ (ou qu'elle est \textit{bijective}) si pour tout $y\in B$, il existe un et \textbf{un seul} $x\in A$ tel que $y=f(x)$.

En d'autre termes, pour tout $y$ de $B$, l'équation $y=f(x)$ n'a qu'une seule solution.

Dans ce cas, il existe une fonction $f^{-1}:B\longrightarrow A$, appelée \textit{réciproque} de $f$, telle que pour tout $x\in A$ et $y\in B$, $y=f(x) \Longleftrightarrow x = f^{-1}(y)$.
\end{mydef}

\begin{thm}
Soit $I$ un intervalle réel et $f:I\longrightarrow\R$ continue. Si de plus $f$ est strictement monotone, alors $f$ établit une bijection de $I$ vers $f(I)$. De plus, sa réciproque $f^{-1}$ est continue sur $f(I)$.
\end{thm}

\begin{exo}[M]
Montrer que pour tout $x>0$, il existe un unique réel $>0$, noté $g(x)$, tel que
\[
g(x)e^{g(x)} = x,
\]
et que la fonction $g:{]0,+\infty[}\longrightarrow{]0,+\infty[}$ ainsi définie est continue.
\end{exo}

\subsection{Fonctions lipschitziennes}

\begin{mydef}[Fonction lipschitzienne]
    On dit que $f$ est \textit{lipschitzienne} s'il existe $M\geq 0$ tel que
    \[
    \forall x,y\in A\quad
    \abs{f(x)-f(y)}\leq M\abs{x-y}.
    \]
    Dans ce cas, $f$ est dite $M$-lipschitzienne.
\end{mydef}

\begin{prop}\label{lipcont}
    Une fonction lipschitzienne est continue.
\end{prop}

\begin{exo}[M]\begin{enumerate}
    \item Montrer la proposition \ref{lipcont}.
    \item Montrer que la réciproque est fausse en considérant la fonction $x\mapsto \frac{1}{x}$ sur $]0,+\infty[$.
\end{enumerate}
\end{exo}

\begin{exo}[Théorème du point fixe de Banach, D]
    Soit $I$ un intervalle réel et $f:I\rightarrow\R$ une fonction $r$-lipschitzienne où $r\in{[0,1[}$. \begin{enumerate}
        \item Montrer que si $f$ admet un point fixe, il est unique.
		\item On suppose que $I=\R$. Montrer que $f$ admet bien un point fixe.\par
		\textit{Indication:} Raisonner par l'absurde: si ce n'est pas le cas, la fonction continue $g:x\longmapsto f(x)-x$ ne s'annule pas sur $\R$.
        \item Soit $u_0\in I$ et $(u_n)$ la suite récurrente de premier terme $u_0$ et de relation de récurrence $u_{n+1}=f(u_n)$. Montrer que $u_n\longrightarrow a$.\par
        \textit{Indication:} Montrer par récurrence que $\forall n\in\N$, $\abs{u_{n+1}-a}\leq r\abs{u_n-a}$.
    \end{enumerate} 
\end{exo}

\begin{exo}[TD]
Soit $f:I\rightarrow\R$ admettant un point fixe $a$. On suppose que $f$ est dérivable en $a$ et que $\abs{f'(a)}<1$. Montrer qu'il existe $\delta>0$ tel que pour tout $u_0\in{]a-\delta,a+\delta[}\cap I$, la suite $u$ de premier terme $u_0$ et vérifiant $u_{n+1}=f(u_n)$ est bien définie et converge vers $a$.\\
\textit{Indication:} Utiliser la définition de la dérivée pour montrer qu'il existe un réel $k>0$ tel que pour tout $n\in\N$, $\abs{u_{n+1}-a}\leq k\abs{u_n-a}$
\end{exo}

\section{Dérivabilité}

\subsection{Théorème de Rolle}

\begin{thm}[de Rolle]
Soit $f$ une fonction continue sur un segment $[a,b]$, dérivable sur son intérieur $]a,b[$. On suppose que $f(a)=f(b)$. Alors, il existe $c\in{]a,b[}$ tel que $f'(c) = 0$.
\end{thm}

\begin{exo}[Théorème de la moyenne de Cauchy, M]
Soient $f$ et $g$ deux fonctions continues sur un segment $[a,b]$, dérivables sur $]a,b[$. Montrer qu'il existe $c\in{]a,b[}$ tel que 
\[
(f(b)-f(a))g'(c) - (g(b)-g(a))f'(c) = 0
\]
\textit{Indication:} Poser $h(x) = (f(b)-f(a))g(x) - (g(b)-g(a))f(x)$, montrer que $h$ est dérivable et s'annule sur $]a,b[$.
\end{exo}

\subsection{Théorème des accroissements finis}

\begin{thm}
Soit $f$ une fonction continue sur un segment $[a,b]$, dérivable sur son intérieur ${]a,b[}$. Alors il existe $c\in{]a,b[}$ tel que
\[
\frac{f(b)-f(a)}{b-a} = f'(c).
\]

Géométriquement, il existe une tangente parallèle à la corde entre $a$ et $b$.
\end{thm}

\begin{cor}[Inégalité des accroissements finis]
Sous les mêmes hypothèses, on a, avec $m$ un minorant de $f'$ et $M$ un majorant de $f'$ sur $]a,b[$ :
\[
m(b-a) \leq f(b) - f(a) \leq M(b-a).
\]

En particulier, si $M$ est un majorant de $\abs{f'}$ sur $]a,b[$, on a
\[
\abs{f(b)-f(a)} \leq M\abs{b-a}.
\]
\end{cor}

\begin{exo}[M]
Soit $f$ une fonction dérivable sur un segment $[a,b]$. On suppose que $f'$ est continue sur $[a,b]$. Montrer que $f$ est lipschitzienne.
\end{exo}

\end{document}