\documentclass[../main.tex]{subfiles}

\begin{document}

\onlyinsubfile{
    \begin{center}
    {\LARGE \textsc{Compléments:\\ Suites, épisode 2}}
    \end{center}
}


\section{Sommes}

\section{Introduction aux séries numériques}

\begin{mydef}[Série]
	Soit $u$ une suite réelle. On appelle \textit{série de terme général $u_n$}, et on note $\sum u_n$, la suite $(S_n)_{n\in\N}$ des \textit{sommes partielles de $(u_n)$} définie par:
	\[
	S_n := u_0 + \cdots + u_n.
	\]
	Le terme $S_n$ est appelé \textit{somme partielle de rang $n$}.
\end{mydef}

\begin{mydef}[Convergence d'une série]\leavevmode
	\begin{itemize}
		\item Soit $\sum u_n$ une série. On dit qu'elle est \textit{convergente} lorsque sa suite des sommes partielles converge \textbf{vers une limite finie}. Dans ce cas, on appelle \textit{somme de la série $\sum u_n$} la limite de $(S_n)$, qu'on note:
		\[
		\sum_{n=0}^{+\infty} u_n = \lim_{n\to +\infty} S_n.
		\]
		\item Dans le cas contraire, la série est dite \textit{divergente} ($(S_n)$ n'admet donc soit pas de limite, soit une limite infinie).
	\end{itemize}
\end{mydef}

\begin{prop}
	Soit $(u_n)$ une suite telle que $\sum u_n$ converge. Alors $u_n\xrightarrow[n\to+\infty]{}0$.
\end{prop}

\begin{exo}[M]
	\begin{enumerate}
		\item Soit $u$ une suite arithmétique de raison $r$ non nulle. Est-ce que la série de terme général $u_n$ converge ?
		\item Soit $q$ un réel. Étudier la convergence de la série $\sum q^n$. Si elle converge, quelle est sa somme ? Sinon, $(S_n)$ admet-elle une limite ?
	\end{enumerate}
\end{exo}

\begin{exo}[M]
	Soit $u$ une suite à termes positifs. Montrer que $\sum u_n$ converge dès que $(S_n)$ est majorée, et que sinon on a $S_n \to +\infty$.
\end{exo}

Ce résultat permet d'ailleurs d'énoncer la propriété suivante:
\begin{prop}
	Soient $u$ et $v$ deux suites positives. On suppose que $u_n\leq v_n$. Alors:
	\begin{itemize}
		\item Si $\sum v_n$ converge alors $\sum u_n$ converge aussi.
		\item Si $\sum u_n$ diverge alors $\sum v_n$ diverge aussi.
	\end{itemize}
\end{prop}

\begin{exo}[Critère de d'Alembert, D]
	Soit $u$ une suite positive. On suppose que \[\frac{u_{n+1}}{u_n}\xrightarrow[n\to+\infty]{}\ell\in\R^+.\] \begin{enumerate}
		\item Montrer que si $0\leq\ell < 1$ alors $\sum u_n$ converge.
		\item Montrer que si $\ell>1$ alors $\sum u_n$ diverge.
		\item Et si $\ell=1$ ?
	\end{enumerate}
	\textit{Indication:} Comparer $u$ et une suite géométrique pour utiliser la proposition précédente.
\end{exo}

\section{Comparaison série-intégrale}

\begin{thm}
	Soit $(u_n)_{n\in\N}$ une suite réelle. On suppose qu'il existe $f:{[0,+\infty[}\rightarrow\R^+$ continue et décroissante telle que pour tout $n\in\N$, $u_n=f(n)$. Alors la série $\sum u_n$ converge ssi la fonction
	\[
	x\in\R^+\longmapsto \int_{0}^{x}f(t)\dif t
	\]
	admet une limite finie en $+\infty$.
\end{thm}

\begin{exo}
	Montrer que la série $\sum \frac{1}{n}$ diverge.
\end{exo}

\end{document}