\documentclass{article}
\usepackage{polyglossia}
\usepackage[a4paper,hmargin=4cm]{geometry}
\usepackage{mathtools, amssymb, amsfonts}
\usepackage{amsthm}
\usepackage{fontspec}
\usepackage{titling}
\usepackage{float}
\usepackage{listings}
\usepackage{graphicx}
\usepackage[usenames,dvipsnames,svgnames]{xcolor}
\usepackage{pgfplots}
\pgfplotsset{compat=newest}
\usepackage{mdframed}
\usepackage[hidelinks]{hyperref}
\usepackage{caption,subcaption}
\usepackage{tikz}
\usetikzlibrary{arrows,positioning,shapes,calc,angles,decorations.markings,patterns,decorations.pathmorphing}
\usepackage[shortlabels]{enumitem}
\usepackage{multicol}
\usepackage{comment}

\setdefaultlanguage{french}
\frenchspacing

\renewcommand{\phi}{\varphi}
\renewcommand{\epsilon}{\varepsilon}
\newcommand{\NN}{\mathbb N}
\newcommand{\ZZ}{\mathbb Z}
\newcommand{\RR}{\mathbb R}
\newcommand{\QQ}{\mathbb Q}
\newcommand{\CC}{\mathbb C}
\newcommand{\KK}{\mathbb K}
\newcommand{\PP}{\mathbb P}
\DeclareMathOperator{\Card}{\mathrm{Card}}
\DeclarePairedDelimiterX{\zint}[2]{[\![}{]\!]}{#1,#2}
\DeclarePairedDelimiterX{\abs}[1]{\lvert}{\rvert}{#1}

%% Titling configuration
\pretitle{\begin{center}\hrulefill\\ \LARGE\textsc}
\posttitle{\vspace{-2mm} \\\hrulefill\end{center}\vspace{-0.5em}}

\title{Exercices de Khôlle -- Terminale S}

\preauthor{\begin{center}}
\author{{\normalfont\scshape Lycée Julie-Victoire Daubié}}
\postauthor{\end{center}}

\date{}

%% Sectioning


\renewcommand{\thesection}{\arabic{section}}

\newcounter{Exercice}
\makeatletter
\newcommand{\exercice}[1][\@nil]{\refstepcounter{Exercice}
	\section*{Exercice \theExercice
	\def\tmp{#1}
	\ifx\tmp\@nnil
	%
	\else
	(#1)
	\fi
}}

\makeatother

\tikzset{roundnode/.style={circle, draw=black!100, thick, minimum size=2mm, fill=white!100,inner sep=0}, dot/.style={draw, circle,fill=black, minimum size=2mm,inner sep=0}}

\tikzset{
	pics/mysymbol/.style={
		code = {
			\def\radius{3cm};
			\draw (0,0) circle (\radius);
			
			\foreach \x in {1,...,#1} {
				\draw[dashed] ({360*(\x + 1)/#1}:\radius) -- (360*\x/#1:\radius);
			}
			
			\foreach \x in {1,...,#1} {
				\node[roundnode] (\x) at (360*\x/#1:\radius) {};
			}
}}}


\begin{document}
	\maketitle

\exercice[Une suite arithmético-géométrique (M)]
Soit $(u_n)_{n\in\NN}$ la suite définie par:
\[
\begin{array}{ll}
u_0 & = 2\\
u_{n+1} &= -2u_n+3
\end{array}
\]



\begin{enumerate}
\item On pose $v_n := u_n - 1$ pour tout $n\in\NN$.
	\begin{enumerate}
	\item Montrer que $(v_n)$ est géométrique ; préciser sa raison et son premier terme.
    \item Déterminer l'expression de $v_n$, puis celle de $u_n$.
	\end{enumerate}
\item On définit la suite $(S_n)$ des sommes de $(u_n)$, par
\[ S_n := \sum_{k=0}^n u_k = u_0+\cdots+u_n. \]
Déterminer l'expression de $S_n$.
\item Étudier la convergence de la suite $(S_n)$ (converge-t-elle ? Si oui, quelle est sa limite ?)
\end{enumerate}
\textbf{Indication} Remarquer que $(u_n)$ est la somme d'une suite géométrique et d'une suite arithmétique, dont on précisera les premiers termes et les raisons respectives.

\exercice[F]
Soit $(u_n)$ une suite réelle, \textbf{supposée géométrique}, telle que
\begin{align*}
u_0 & = 3\\
u_3 &= \frac 3 {16\sqrt 2}.
\end{align*}

\begin{enumerate}
\item Déterminer la raison de $(u_n)$.
\item On pose, pour tout $n\in\NN$,
\[ S_n := \sum_{k=0}^n u_k = u_0 + \cdots + u_n. \]
Déterminer l'expression de $S_n$. Étudier la convergence de $(S_n)$ et, si elle existe, déterminer sa limite.
\end{enumerate}


\exercice[M]

Soit $(u_n)$ la suite \textbf{complexe} définie par:
\begin{align*}
u_0 &= i\\
u_{n+1} &= \left(\frac 1 4 + i\frac{\sqrt{3}}{4}\right)u_n
\end{align*}

\begin{enumerate}
\item De quel type de suite $\left(
\abs{u_n}\right)_{n\in\NN}$ est-elle ? Quelle est sa raison ?
\item Étudier la convergence de $\left(\abs{u_n}\right)$ (converge-t-elle ? Si oui, que vaut sa limite ?)
\item Même questions que la 1. pour la suite $\left(\arg(u_n)\right)_{n\in\NN}$. Déterminer son expression et étudier sa convergence.
\end{enumerate}

\exercice[F]
Déterminer les limites des suites suivantes:
\begin{multicols}{2}
\begin{enumerate}
\item $u_n = \frac{\sin(n)}{n}$
\item $u_n = \dfrac{\ln(n)}{\sqrt n} $
\item $u_n = \ln(n+1) -\ln(n) $
\item $u_n = \sqrt{n+3} - \sqrt{n+2} + \sqrt{n+1} - \sqrt{n} $
\item $u_n = \dfrac{1}{n+\cos(n)}$
\item $u_n = n\ln\left(1+\dfrac 1 n\right) $\\ \textbf{Indication} Quelle est la limite de la fonction $x\longmapsto \dfrac{\ln(1+x)}{x}$ en 0 ?
\item $u_n = n\sin\left(\dfrac 1 n\right)$\\ \textbf{Indication} Quelle est la limite de la fonction $x\longmapsto \dfrac{\sin(x)}{x}$ en 0 ?
\end{enumerate}
\end{multicols}


\exercice[Convergence d'une série (D)]
Le but de l'exercice est de démontrer la convergence de la suite $(S_n)$ définie pour tout $n\geq 0$ par
\[
S_n := \sum_{k=0}^n \frac{\ln(1+k)}{3^k} = \frac{\ln(1+0)}{3^0} + \frac{\ln(1+1)}{3^1} + \cdots + \frac{\ln(1+n)}{3^n}.
\]
\begin{enumerate}
\item Montrer que la suite $(S_n)$ est croissante.\\
\textbf{Indication} Que vaut $S_{n+1} - S_n$ ?
\item Montrer par récurrence la propriété suivante pour tout entier naturel $n$:
\begin{equation}
\tag{$P_n$}
n < 2^n
\label{prop:PnConv}
\end{equation}
\item Montrer que pour tout réel $x\geq 0$,
\[ \ln(1+x) \leq x. \]
On montrera que l'application $f\colon\RR^+\rightarrow\RR$ définie par
\[ f(x) = \ln(1+x) - x \]
est positive, via une étude de fonction.
\item En déduire que pour tout $k\geq 0$, $\ln(1+k) < 2^k$, puis que
\[ \frac{\ln(1+k)}{3^k} < \left(\frac{2}{3}\right)^k \]
\item En déduire que
\[ 0\leq S_n \leq \sum_{k=0}^n \left(\frac{2}{3}\right)^k = 1 + \left(\frac{2}{3}\right)^1 + \cdots + \left(\frac{2}{3}\right)^n. \]
\item Conclure.
\end{enumerate}

\exercice[Marches aléatoires (M)]
On considère une grille $n\times n$, où $n\in\NN^*$ de points du plan $(k,\ell)$ pour $0\leq k\leq n-1$ et $0\leq\ell\leq n-1$.
\begin{figure}[h]
\centering
\begin{tikzpicture}
\draw[->,thick] (-1,0) -- (8,0) node[pos=1,label=below:$x$] {};
\draw[->,thick] (0,-1) -- (0,8) node[pos=1,label=left:$y$] {};

\foreach \x in {0,...,7}
{
	\foreach \y in {0,...,7} {
		\node[roundnode] at (\x,\y) {};
}}


\end{tikzpicture}
\end{figure}

On étudie les chemins construits en faisant des pas vers la droite ou vers le haut, où chaque direction à une probabilité d'être choisie de $\frac 1 2$ à chaque étape. On appelle \textit{marche aléatoire} un tel chemin.

On considère 

Soit $X$ la variable aléatoire égale au nombre de déplacements vers la droite.

\begin{enumerate}
\item Déterminer la loi de $X$.
\item Combien de pas la droite fait-on en moyenne ?
\end{enumerate}


\exercice[F]
Soit $(\Omega,\mathcal A,\PP)$ un espace probabilisé (un ensemble muni d'une probabilité). Calculer $\PP(B)$:
\begin{enumerate}
\item Sachant que $\PP(A) = 1/3$ et $\PP_A(B) = 2/5$ et $\PP_{\overline A}(B)=1/4$.
\item Sachant que $\PP(A) = 2/3$, $\PP(A\cap B)=1/3$ et $\PP(\overline A\cap B) = 1/6$.
\end{enumerate}


\exercice[M]
Pour tout entier naturel $n$ supérieur ou égal à 2, on note $\mathbb U_n$ l'ensemble des nombres complexes $a$ tels que $a^n=1$. On montre que ses éléments sont de la forme
\[
\omega^k = e^{2ik\pi/n},\ 0\leq k\leq n-1,\ \text{où}\ \omega = e^{2i\pi/n}.
\]


\begin{figure}[H]
	\centering
\begin{subfigure}[b]{0.45\textwidth}
	\centering
	\resizebox{\linewidth}{!}{
	\begin{tikzpicture}
	\node[dot,label=above right:$O$] at (0,0) {};
	
	\draw[thick,->] (-4,0) -- (4,0) node[pos=1,anchor=south east] {$x$};
	\draw[thick,->] (0,-4) -- (0,4) node[pos=1,anchor=north west] {$y$};
	
	\path pic at (0,0) {mysymbol={3}};
	\end{tikzpicture}
	}
	\subcaption{$\mathbb U_3$.}
\end{subfigure}
\begin{subfigure}[b]{0.45\textwidth}
	\centering
	\resizebox{\linewidth}{!}{
	\begin{tikzpicture}
		\node[dot,label=above right:$O$] at (0,0) {};
		
		\draw[thick,->] (-4,0) -- (4,0) node[pos=1,anchor=south east] {$x$};
		\draw[thick,->] (0,-4) -- (0,4) node[pos=1,anchor=north west] {$y$};
		
		\path pic at (0,0) {mysymbol={5}};
	\end{tikzpicture}
	}
	\subcaption{$\mathbb{U}_5$.}
\end{subfigure}
\begin{subfigure}[b]{0.45\textwidth}
	\centering
	\resizebox{\linewidth}{!}{
		\begin{tikzpicture}
		\node[dot,label=above right:$O$] at (0,0) {};
		
		\draw[thick,->] (-4,0) -- (4,0) node[pos=1,anchor=south east] {$x$};
		\draw[thick,->] (0,-4) -- (0,4) node[pos=1,anchor=north west] {$y$};
		
		\path pic at (0,0) {mysymbol={8}};
		\end{tikzpicture}
	}
	\subcaption{$\mathbb{U}_{8}$.}
\end{subfigure}
\begin{subfigure}[b]{0.45\textwidth}
	\centering
	\resizebox{\linewidth}{!}{
		\begin{tikzpicture}
		\node[dot,label=above right:$O$] at (0,0) {};
		
		\draw[thick,->] (-4,0) -- (4,0) node[pos=1,anchor=south east] {$x$};
		\draw[thick,->] (0,-4) -- (0,4) node[pos=1,anchor=north west] {$y$};
		
		\path pic at (0,0) {mysymbol={12}};
		\end{tikzpicture}
	}
	\subcaption{$\mathbb{U}_{12}$.}
\end{subfigure}
\end{figure}

On munit $\mathbb U_n$ de la mesure de probabilité uniforme $\PP$. Ainsi, pour tout $\omega\in\mathbb U_n$,
\[ \PP(\{\omega\}) = \frac{1}{n}. \]

\begin{enumerate}
	\item On prend au hasard un élément $z$ de $\mathbb U_n$. Quelle est l'espérance mathématique de la variable aléatoire $Z$ égale à $z$ ?
	\item Calculer l'espérance mathématique du module et de l'argument de $Z$.
	\item On prend deux éléments $w$ et $z$ de $\mathbb U_n$ avec une probabilité uniforme; on note $M_w$ et $M_z$ les points du plan complexe dont ils sont les affixes.
	Quelle est la valeur moyenne de l'angle $\angle(M_wOM_z)$ ?
\end{enumerate}

\exercice[F]
Résoudre les équations suivantes dans le corps $\CC$ des nombres complexes, calculer le module et l'argument des solutions, et les placer dans le plan complexe ramené à un repère orthonormé $(O,\vec u,\vec v)$:
\begin{enumerate}
\item $z^2+z+1 = 0$
\item $z^2+4z+6 = 0$
\item $z^4 + 6z^2 + 25 = 0$
\end{enumerate}

\exercice[D]
Soit $f\colon\CC\longrightarrow\CC$ définie pour tout $z\in\CC$ par:
\[ f(z) = z^3+5z^2+7z-13. \]
On cherche à factoriser $f$ sur $\RR$, puis sur $\CC$.
\begin{enumerate}
\item Vérifier que $1$ est racine de $f$, c'est-à-dire que $f(1) = 0$.
\item D'après un théorème que l'on admettra (cf. cours de L1 de mathématiques), il existe alors un trinôme du second degré à coefficients réels,
\[ q(z) = az^2+bz+c \]
tel que
\[ f(z) = (z-1)q(z). \]
\begin{enumerate}
\item Déterminer $a$, $b$, et $c$ pour que l'égalité précédente soit vraie.
\item Factoriser $q(z)$ (donc résoudre $q(z) = 0$). En déduire la forme factorisée de $f(z)$.
\end{enumerate}
\end{enumerate}

\exercice[Bordeaux 1980]
Pour tout réel $a$, on définit sur $\RR$ la fonction numérique $f_a$ par \[f_a(x) = e^{-x} + ax.\]
Soit $\mathcal C_a$ sa représentation graphique dans le plan équipé d'un repère orthonormé $(O,\vec{\imath},\vec{\jmath})$.

\begin{enumerate}
	\item Étudier les variations de $f_a$. Pour quelles valeurs de $a$ la fonction $f_a$ admet-elle un extremum ? On appelle $A$ l'ensemble de ces valeurs.
	\item Pour tout tel réel $a\in A$, on appelle $M_a$ le point de $\mathcal C_a$ correspondant à l'extremum. Déterminer ses coordonnées.
	\item Montrer que l'ensemble $E$ des points $M_a$, lorsque $a$ décrit $A$, est la courbe représentative d'une fonction $g$. Étudier les variations de $g$, et dessiner $E$.
\end{enumerate}


\exercice[Paris 1980]

Soit la famille d'équations
\begin{equation}\tag{$E_\theta$}
	z^2\sin^2\theta - 4z\sin\theta + 4 + \cos^2\theta = 0
	\label{EqTheta}
\end{equation}
où $\theta\in{\left]0,\pi\right[}$.

\begin{enumerate}
	\item Soit $0< \theta< \pi$. Résoudre l'équation \eqref{EqTheta} dans le corps $\CC$ des nombres complexes. On note $z_1$ et $z_2$ les racines.
	\item On note $M_1$ et $M_2$ dont les nombres complexes $z_1$ et $z_2$ sont les affixes dans un repère orthonormé $(O,\vec u,\vec v)$.\\
	Dessiner l'ensemble des points $M_1$ et $M_2$ lorsque ${]0,\pi[}$.
\end{enumerate}


\exercice[Pondichéry 1996]

Un fabricant de berlingots possède trois machines A, B et C qui fournissent respectivement 10 \%, 40 \% et 50 \% de la production totale de son usine.
Une étude a montré que le pourcentage de berlingots défectueux est de 3,5 \% pour la machine A, de 1,5 \% pour la machine B et de 2,2 \% pour la machine C.
Après fabrication, les berlingots sont versés dans un bac commun aux trois machines.
On choisit au hasard un berlingot dans le bac.

\begin{enumerate}
	\item Montrer que la probabilité que ce berlingot provienne de la machine C et soit défectueux est $0{,}011$.
	\item Calculer la probabilité que ce berlingot soit défectueux.
	\item Calculer la probabilité que ce berlingot provienne de la machine C sachant qu'il est défectueux.
	\item On prélève successivement dans le bac 10 berlingots en remettant à chaque fois le berlingot tiré dans le bac. Calculer la probabilité d'obtenir au moins un berlingot défectueux parmi ces 10 prélèvements.
\end{enumerate}


\exercice[Amérique du Nord 1996]

On désigne par $n$ un entier naturel $\geq 2$. On se donne $n$ sacs de jetons $S_1,\ldots,S_n$. Au départ, le sac $S_1$ contient 2 jetons noirs et 1 jeton blanc, et chacun des autres sacs contient 1 jeton noir et 1 jeton blanc.

On se propose d'étudier l'évolution des tirages successifs d'un jeton de ces sacs, effectuées de la façon suivante:
\begin{itemize}
	\item \textbf{Première étape} on tire au hasard un jeton de $S_1$,
	\item \textbf{Deuxième étape} on place ce jeton de $S_2$ et on tire, au hasard, un jeton de $S_2$,
	\item \textbf{Troisième étape} après avoir placé dans $S_3$ le jeton sorti de $S_2$ on tire, au hasard, un jeton de $S_3$ et ainsi de suite. 
\end{itemize}

\begin{figure}[h]
	\centering
\begin{tikzpicture}
\tikzset{ball/.style={circle, draw=black!100, thick, minimum size=1mm}}
\draw[thick] (-3,1) -- (-3,0) -- (-1,0) -- (-1,1) node[pos=0.2,label=below:$S_1$] {};
\node[ball,fill=black] at (-2.8,0.2) {};
\node[ball,fill=black] at (-2.4,0.2) {};
\node[ball,fill=white] at (-1.9,0.2) {};

\path[thick,out=90,in=100,->] (-1.1,1.2) edge (0.1,1.2);

\draw[thick] (0,1) -- (0,0) -- (2,0) -- (2,1) node[pos=0.2,label=below:$S_2$] {};
\node[ball,fill=black] at (0.25,0.2) {};
\node[ball,fill=white] at (0.7,0.2) {};

\path[thick,out=90,in=100,->] (1.9,1.2) edge (3.1,1.2);

\draw[thick] (3,1) -- (3,0) -- (5,0) -- (5,1) node[pos=0.2,label=below:$S_3$] {};
\node[ball,fill=white] at (3.4,0.2) {};
\node[ball,fill=black] at (3.85,0.2) {};
\end{tikzpicture}
\end{figure}

Pour tout $k\in\NN$ tel que $1\leq k\leq n$, on note $E_k$ l'événement << le jeton sorti de $S_k$ est blanc >> ; on notera classiquement $\overline{E_k}$ son évènement contraire.

\begin{enumerate}
	\item \begin{enumerate}
		\item Déterminer la probabilité de $E_1$ et les probabilités conditionnelles $\PP(E_2|E_1)$ et $\PP(E_2|\overline{E_1})$.\\
		En déduire la probabilité de $E_2$.
		\item Pour tout entier $1\leq k\leq n$, on note la probabilité $\PP(E_k) = p_k$.\\
		Démontrer le relation de récurrence
		\[ p_{k+1} = \frac 1 3 p_k + \frac 1 3. \]
	\end{enumerate}
	\item On définit $(u_k)_{k\geq 1}$ une suite réelle telle que $u_1= \frac 1 3$, telle que 
	\[ u_{k+1} = \frac 1 3 u_k + \frac 1 3.\]
	On pose alors $v_k := u_k - \frac 1 2$.
	\begin{enumerate}
		\item Montrer que la suite $(v_k)$ est géométrique.
		\item En déduire l'expression de $u_k$. Étudier la convergence de la suite $(u_k)$.
	\end{enumerate}
	\item On suppose que $n=10$. Déterminer pour quelles valeurs de $k$ on a
	\[ 0{,}4999\leq p \leq 0{,}5. \]
\end{enumerate}

\exercice[Amiens 1990 (1)]

Soit $f\colon\CC\rightarrow\CC$ l'application définie par
\[ f(z) = z^4 - \sqrt 2 z^3 - 4\sqrt 2 z - 16. \]

\begin{enumerate}
	\item Calculer $f(2i)$ et $f(-2i)$.
	\item D'après un théorème que l'on admettra, il existe un trinôme du second degré à coefficients réels $q(z)= z^2+az+b$ tel que
	\[ f(z) = (z^2+4)q(z). \]
	Trouver $a$ et $b$.
	\item En déduire les solutions de l'équation $f(z) = 0$ sur $\CC$.
	\item Placer dans le plan complexe rapporté à un repère orthonormé $(O,\vec u,\vec v)$ les points $A,B,C,D$ qui ont pour affixes les solutions de la question précédente.
	\item Montrer que $A,B,C,D$ appartiennent à un même cercle $(\mathcal{C})$ dont on précisera le centre et le rayon.
\end{enumerate}


\exercice[Amiens 1990 (2)]

Pour tout $k > 0$, on considère la fonction $f_k$ définie sur $]0,+\infty[$ par
\[ f_k(x) = k^2x^2 - \frac 1 4 - \frac 1 2 \ln x. \]
On note $\mathcal{C}_k$ sa courbe représentative dans le plan muni d'un repère orthonormé $(O,\vec\imath,\vec{\jmath})$.
\begin{enumerate}
	\item Étudier les variations de $f_k$, dresser son tableau. On précisera les limites de $f_k$.
	\item Soit $M_k$ le point $\mathcal{C}_k$ correspondant au minimum de $f_k$. Déterminer ses coordonnées.
	\item Déterminer dans le repère $(O,\vec{\imath},\vec{\jmath})$ une équation cartésienne $y=g(x) $ vérifiée par l'ensemble $\mathcal{A}$ des points $M_k$, $k >0 $.
	\item Préciser la position relative de $\mathcal{C}_k$ et $\mathcal{A}$. Tracer $\mathcal{C}_1$ et $\mathcal{A}$. 
\end{enumerate}

\exercice[Amérique du Nord 1986]

Soit, pour tout $z\in\CC$ le polynôme
\[ P_\lambda(z) = z^2 - 4z + \lambda \]
où $\lambda\in\RR$.

\begin{enumerate}
	\item Montrer que si $P_\lambda(z) = 0$ admet une racine $z_\lambda$ alors $\overline{z_\lambda}$ est aussi solution.
	\item Montrer que l'équation $P_\lambda(z) = 0$ admet au moins une solution réelle.
	\item Déterminer $\lambda$ pour que l'équation $P_\lambda(z)=0$ admette au moins une racine réelle de module égal à 2. Résoudre l'équation pour cette valeur de $\lambda$.
	\item Déterminer $\lambda$ pour que $P_\lambda(z)=0$ admette une racine \textbf{complexe} de module égal à 2. Résoudre l'équation pour les valeurs de $\lambda$ trouvées, préciser le module et l'argument de chaque solution.
\end{enumerate}

\exercice[Amérique du Sud 1986]

Soit $P$ le plan complexe muni d'un repère orthonormé $(O,\vec u,\vec v)$. Au points $M(x,y)$ on associe, classiquement, son affixe $z = x+iy$.\\
Soient $A$ et $B$ les points d'affixes respectives $1+i$ et $-3$.

À un point $M$, distinct de $A$ ou $B$ et d'affixe $z$, on associe le(s) point(s) $M'$, s'ils existent, d'affixes $z'$ telles que
\[
\left\{
\begin{array}{ll}
\dfrac{z'+3}{z+3} &\text{imaginaire pur} \\
\dfrac{z'-1-i}{z-1-i} &\text{réel}.
\end{array}
\right.
\]

\begin{enumerate}
	\item Donner un sens géométrique à $\arg\left(\dfrac{z'+3}{z+3}\right)$ et $\arg\left(\dfrac{z'-1-i}{z-1-i}\right)$.
	\item Démontrer géométriquement qu'il existe un cercle $\mathcal{C}$ du plan tel que si $M\in P\setminus \mathcal{C}$, alors $M'$ existe et est unique. Construire alors $M'$.
\end{enumerate}


\exercice[Bordeaux 1984]

Soit $\theta\in[0,2\pi]$.
\begin{enumerate}
	\item Résoudre dans $\CC$ l'équation
	\[ z^2 - (2^{\theta+1}\cos\theta)z + 2^{2\theta} = 0, \]
	et donner chaque solution sous forme trigonométrique.
	\item Le plan étant rapporté à un repère orthonormé $(O,\vec u,\vec v)$, on considère les points $A$ et $B$ dont les affixes sont les solutions de l'équation précédente. Déterminer $\theta$ de manière à ce que le triangle $OAB$ soit équilatéral.
\end{enumerate}


\exercice[Montpellier 1984]

On ramène la plan à un repère orthonormé $(O,\vec \imath,\vec \jmath)$. Soit $f$ l'application définie sur $\RR$ par
\[ 
\begin{cases}
f(x) = x \ln\left(1+\frac 1 x\right) &\forall x > 0 \\
f(0) = 0. &
\end{cases}
 \]

\begin{enumerate}
	\item Étudier la continuité et la dérivabilité de $f$ en $0$.
	\item On considère la fonction $g$ définie sur $[1,+\infty[$ par
	\[ g(x) = x\ln x \]
	et on appelle $\Gamma$ sa courbe représentative. Étudier $g$ et tracer $\Gamma$.
	\item Étudier la limite $f$ en $+\infty$. Montrer que les courbes $\Gamma$ et $\mathcal{C}$ sont asymptotes l'une de l'autre et préciser leur positions relatives.\\
	\textbf{Rappel} Dire que les deux courbes sont asymptotes revient à dire que \[\lim\limits_{x\to+\infty}\left(f(x)-g(x)\right)=0.\]
	\item Montrer que $f$ est deux fois dérivable, calculer sa dérivée $f'$ et sa dérivée seconde $f''$ (la dérivée de sa dérivée). Étudier les variations de $f'$ et montrer qu'elle est positive.
	\item Achever l'étude de la fonction $f$. Tracer la courbe $\mathcal{C}$ sur la même figure que $\Gamma$.
\end{enumerate}


\exercice[Dijon 1982]

$n$ étant un entier naturel fixé, on considère l'équation dans $\ZZ^2$
\begin{equation}\tag{$E_n$}
	165x - 132y = n
\end{equation}

Résoudre cette équation pour:
\begin{enumerate}
	\item $n=0$.
	\item $n=33$.
	\item $n=66$.
	\item $n=42$.
\end{enumerate}
Dans chaque cas, on déterminera non seulement le couple de solutions $(x,y)$ mais aussi leur PGCD.

\exercice[Un peu d'optique (D)]

Une lame à face parallèles, d'épaisseur $e$ et d'indice de réfraction $n > 1$ est éclairée par un rayon laser de longueur d'onde $\lambda$, en incidence normale.

\begin{figure}
	\centering
	\begin{tikzpicture}[scale=1.5]
	\tikzset{->-/.style={decoration={
		markings,
		mark=at position #1 with {\arrow[>=stealth]{>}}},postaction={decorate}}}
	
	\draw[draw=black!80,semithick,fill=gray!50] (-1.5,1.5) rectangle (0,-1) node[pos=0.003, anchor=south west] {Lame de verre};
	\node[circle,draw=black,anchor=north west,minimum size=1pt,inner sep=1mm] at (-1.35,1.35) {$n$};
	
	\draw[->-=0.5,red,semithick] (-2.5,0.9) -- (-1.5,0.9) node[pos=0,anchor=south] {Laser};
	
	\def\cr{0.91}
	
	\pgfmathsetmacro{\mycol}{100}
	\edef\mycol{red!\mycol}
	\draw[->-=0.5,\mycol,semithick] (-1.5,0.9) -- (0,0.9);
	
	\pgfmathsetmacro{\mycol}{100*\cr^1}
	\edef\mycol{red!\mycol}
	\draw[->-=0.5,\mycol,semithick] (0,.9) -- (1,.9) node[pos=1,black,anchor=west] {$(1)$};
	
	\def\st{0.26}
	
	\foreach \z in {1,...,6} {
		\def\alt{0.9+\st/2-\st*\z}
		\pgfmathsetmacro{\mycol}{{100*\cr^\z}}
		\edef\mycol{red!\mycol}
		\draw[->-=0.5,\mycol,semithick] (0,\alt) -- (-1.5,\alt);
		
		\pgfmathsetmacro{\mycol}{{100*\cr^(\z+0.5)}}
		\edef\mycol{red!\mycol}
		\draw[->-=0.5,\mycol,semithick] (-1.5,{\alt-\st/2}) -- (0,{\alt-\st/2});
		
		\pgfmathtruncatemacro\result{\z + 1}
		\draw[->-=0.5,\mycol,semithick] (0,{\alt-\st/2}) -- (1,{\alt-\st/2}) node[pos=1,black,anchor=west] {$(\result)$};
	}
	\end{tikzpicture}
\end{figure}

Le laser est une onde électromagnétique, dont l'amplitude et la phase sont représentées par un nombre complexe $A_0$, où
\begin{equation*}
	\left\{
	\begin{array}{ll}
	\abs{A_0} &\text{est l'amplitude de l'onde}\\
	\arg(A_0) &\text{est la phase de l'onde}
	\end{array}\right.
\end{equation*}

On suppose qu'à chaque fois que le rayon rencontre la surface entre la lame et l'air, il est divisé en deux: un rayon transmis et un rayon réfléchi. 

Ainsi, le rayon entrant fait des allers-retours dans la lame, ricochant sur la surface verre/air en laissant passer à chaque fois une partie de la lumière.

L'amplitude de l'onde réfléchie est multipliée par un réel $0 < r < 1$. L'amplitude de l'onde transmise est multipliée par un réel $0< t < 1$.

\begin{enumerate}
	\item Quelle est la différence de marche $\delta$ entre deux rayons successifs ?
	\item En déduire l'expression du déphasage $\phi = 2\pi\delta/\lambda$ entre deux rayons successifs.
	\\
	On admet alors que la relation entre les phases de deux rayons successifs $(p)$ et $(p+1)$ est
	\[ \arg(A_{p+1}) = \arg(A_p) + \phi. \]
	Exprimer $\arg(A_p)$ en fonction de $\phi$ et $p$.
	\item Combien de réflexions le rayon $(p+1)$ subit-il par rapport au rayon $(p)$ ? En déduire une relation entre $\abs{A_{p+1}}$ et $\abs{A_p}$.
	\item Exprimer $\abs{A_p}$ en fonction de $r$ et $p$.
	\item En déduire l'expression de $A_p$ en fonction de $r$, $\phi$ et $p$.
	\item On définit l'amplitude de l'onde créée par les rayons de $(1)$ à $(p)$,
		\[ S_p = \sum_{k=1}^{p}A_k = A_1 + \cdots + A_p. \]
	Déduire des questions précédentes que l'expression de $S_p$ en fonction de $\phi$, $r$ et $p$ est
		\[ S_p = A_1\frac{1-r^{2p}e^{ip\phi}}{1-r^2e^{i\phi}}. \]
	\item Le rayon $(1)$ est issu du rayon initial d'amplitude $A_0$ après deux transmissions. Justifier qu'alors $A_1 = t^2A_0$, et en déduire l'expression de $S_p$ en foncton de $\phi$, $r$, $t$ et $p$.
	\\
	On admet que $\lim\limits_{p\to+\infty}r^{2p}e^{ip\phi} = 0$. On a alors que l'amplitude de l'onde totale est
		\[ S = \lim_{p\to+\infty}S_p = t^2A_0\frac{1}{1-r^2e^{i\phi}}. \]
	\item Exprimer $I(\phi) = \abs{S}^2$, l'intensité lumineuse en sortie de la lame, en fonction de $t$, $r$, $A_0$ et $\phi$.
	\item Pour quelles valeurs de $\phi$ la fonction $I$ est elle maximale ?
\end{enumerate}

\begin{figure}
	\centering
	\begin{tikzpicture}
	\begin{axis}[samples=500,
		xscale=1.6,
		domain={1/7*10^7}:{1/5*10^7},
		axis lines=middle,
		axis line style={->},
		ymax = 1,
		xlabel near ticks,
		ylabel near ticks,
		xlabel={Nombre d'onde $\frac 1 \lambda$},
		ylabel={Transmittance $I(\phi)/I_0$}
		]
	\addplot[blue]plot (\x,{1/(1+100*sin(2*pi*1.33*0.0001*\x)^2)});
	\end{axis}
	\end{tikzpicture}
	\caption{Graphe de la \textit{transmittance} $T$, rapport entre l'intensité lumineuse sortante $I(\phi)$ et l'intensité entrante $I_0$.}
\end{figure}


\end{document}
